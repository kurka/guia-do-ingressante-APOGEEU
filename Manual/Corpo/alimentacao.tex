% Este arquivo tex vai ser incluído no arquivo tex principal, não pe preciso
% declarar nenhum cabeçalho

\begin{story}{Comida}

\begin{sidebar}{Como carregar o cartão?}

Simples: vá ao guichê ao lado direito da entrada do Bandejão, e faça, por exemplo, um depósito de R\$20,00 para 10 créditos. Fique atento para o horário que o guichê fica aberto, porque não é necessariamente o mesmo do bandejão (e muda regularmente, nem adiantaria escrever aqui). Outra maneira de colocar créditos no RA é fazer um depósito na conta do bandejão no Santander (Ag.: 207 / Conta: 43.010.009-2) ou no Banco do Brasil (Ag.: 4203-X / Conta: 66.315-8) e depois carregar o seu cartão, no guichê do bandejão ou na Central de Informações (próximo à Reitoria). Não são aceitos comprovantes de pagamento de entrega de envelope ou via internet. Fique esperto para não ir ao RA ou sem créditos, pois lá é o único restaurante onde não é possível carregar o cartão.

Os bandejões funcionam de segunda a sexta, nos seguintes horários:

\begin{itemize}
\item RU: das 10h30 às 14h00 (almoço) e das 17h30 às 19h45 (jantar)
\item RA: das 11h30 às 14h00 (almoço) e das 17h30 às 19h (jantar)
\item RS: das 11h30 às 14h00 (almoço) e fechado para jantar
\end{itemize}

\end{sidebar}

\section*{Dia-a-dia}
\addcontentsline{toc}{section}{Dia-a-dia}

\subsection*{Bandejão}
\addcontentsline{toc}{subsection}{Bandejão}

Um dos momentos de glória do dia de um futuro mestre ou doutor é O Bandejão. É a hora de intensas e indiscutíveis emoções. Caso sua salada corra sobre a mesa, mantenha-se calmo. Evite discussões e jamais tente descobrir o sabor do suco pelo paladar (limão ou pêssego?). É mais cômodo ler no cardápio do dia. Uma dica: para cortar o bife, comece fazendo muita força e quando começar a amolecer pare, porque você chegou na bandeja.

Falando sério agora: o Bandejão (Restaurante Universitário), ou simplesmente Bandex, fica ao lado da Biblioteca Central, bem em frente ao PB (Prédio Básico, ou Ciclo Básico II) e, a menos que você não queira economizar uma boa grana com comida, vai ser o lugar onde você vai estar na maioria dos seus horários de almoço. Com o tempo, você vai ver que o Bandejão é o ``coração da Unicamp'': o local de você se encontrar com os amigos (combinando ou não antes), contar os micos nas aulas, jogar conversa fora, e falar mal da comida, que nem é tão ruim assim como muitos dizem. Sem dúvida, é o melhor custo-benefício da Unicamp: por R\$2,00, você tem direito a arroz, feijão, salada, proteína de soja, suco, chá e café à vontade. A carne e a sobremesa tem que dar uma choradinha para a tiazinha para poder repetir, mas geralmente elas deixam.

Em todo o caso, se você fica bastante tempo no seu laboratório, é muito provável que você coma mais regularmente no RA (Refeitório da Administração), também conhecido como Pratex, pelo fato de a comida ser servida em pratos, e não em bandejas. Ele fica atrás da FEEC, perto do prédio da Engenharia Básica. Tem algumas diferenças em relação ao Bandejão: o espaço físico é bem menor, por exemplo. No RA você mesmo se serve, apesar de a carne às vezes ser servida pela mulher que trabalha lá. Se você for com um amigo, vá com paciência para esperar, porque é difícil pra arrumar lugar, além ser relativamente apertado.

Há também um outro restaurante universitário, mais novo, localizado perto do Instituto de Computação e da Faculdade de Engenharia Civil (um prédio azul bem grande, no alto do campus), que é o chamado RS (Restaurante Universitário da Rua Saturnino). Ele fica na rua Saturnino de Brito, 314 e tem o diferencial de servir uma opção de \textbf{cardápio vegetariano}, porém só abre durante o almoço. Para poder usar o Bandejão, o RA ou o RS, você deve estar com o seu cartão universitário carregado.

Para saber previamente o cárdápio do Bandejão acesse-o no site da prefeitura:

\begin{itemize}
\item \url{www.prefeitura.unicamp.br/servicos.php?servID=119}
\end{itemize}

Outra opção é o Bandecowap (\url{tinyurl.com/bandecowap}). Ele foi feito para consultar o cardápio nos celulares mais simples que ofereceminternet na forma de WAP. Para Android, estão disponíveis os aplicativos BandecoDroid Unicamp, para consulta de cardápio, e Unicamp Serviços, do CCUEC, que informa cardápio e saldo no seu cartão, entre outros serviços.

\subsection*{Outros lugares para as refeições}
\addcontentsline{toc}{subsection}{Outros lugares para as refeições}

\subsubsection*{Na Unicamp}

Alguns lugares que servem pratos feitos são a cantina da Física (um dos melhores da Unicamp, serve também meio-prato) e a da Química (bem parecido com o da Física). A Física também serve \emph{self-service}, mas é meio caro. Outros lugares que servem comida por quilo são o DCE (um dos mais baratos na Unicamp), a Biologia (comida gostosa e não tão cara quanto a Física), a Educação e a Mecânica (a mais perto da FEEC). Por fim, se você é vegetariano, uma boa dica é o Gatti (que fica do lado do Instituto de Computação e da Economia, no pavilhão das Cênicas/Dança).

\subsubsection*{Fora da Unicamp}

Próximo ao balão da avenida 1, temos também o Terraço, que vende marmitex e tem self-service a um preço bom, além de churrasco às terças e quintas. Um pouco mais acima na avenida 1, tem o Bardana (um com a fachada toda vermelha), que tem a mesma faixa de preço que o Terraço, e costuma ser considerado bem melhor e com uma excelente variedade, tanto de saladas quanto de carnes. Próximo ao Bardana, há o Pepe Loco, que serve comida mexicana no estilo fast-food. Na frente da reitoria há o Del Sol, o Ginza e o Moriá. O Del Sol serve comida por quilo, sendo parecido (em preço e pratos) com o Bardana, enquanto que o Ginza serve à la carte com preços bons (uma dica é a feijoada completa às quartas, que sai por R\$10,00 e inclui uma mini-capirinha!) e o Moriá serve pratos feitos a preços mais baratos. Próximo ao Ginza, em frente à guarita do HC, há o Campus Grill, com comida boa a um preço um tanto alto (um pouco mais caro que a cantina da Física). Na avenida 2, próximo ao balão, há o Aulus, que é o mais caro dos citados aqui, mas é muito bom e com uma decoração bem excêntrica.

\subsection*{Lanches e sucos}
\addcontentsline{toc}{subsection}{Lanches e sucos}

Está de tarde, bateu fome e quer comer um lanche (hambúrguer, pão-na-chapa, queijo quente, x-salada, croissant, qualquer coisa do gênero)? Quase todas as cantinas da Unicamp servem lanches. Algumas boas são a Física e a Química.

Quase todas as cantinas servem salgados prontos, lanches naturais e coisas do gênero. Para sucos, há três lugares muito bons: a cantina da Física, a da Química e a famosíssima banca de sucos do CB, que tem milhões de sucos, vende frutas e também salgados. Todo dia a banca de sucos do CB tem um sabor na oferta, que é ótimo pra sair do tradicional suco de laranja.

Nas quartas-feiras, desde a manhã até depois do almoço, há uma feira no centro da praça do CB, na qual há opções bem variadas, desde doces  e açaí a pastéis e comida japonesa, embora geralmente mais caras que as cantinas. Algumas das barracas abrem também na quinta-feira.

\subsection*{Padarias e café da manhã}
\addcontentsline{toc}{subsection}{Padarias e café da manhã}

Quatro cantinas da Unicamp abrem bem cedo e servem o bom pingado com pão na chapa matinal. São elas a da Mecânica, a do DCE, a da Química e a da Física.

A Padaria Alemã serve uma bandeja de café da manhã que custa por volta de dez reais com suco, café-com-leite/chocolate, croissant, mamão, bolo, pão francês, torradas, manteiga e geléia. Ainda há a possibilidade de fazer trocas como: suco por chocolate, croissant por dois pães-na-chapa, mamão por banana e coisas do gênero. Também são servidos lanches gigantescos, com muitas opções de recheio, por um preço relativamente barato (cerca de R\$9,00), então tenha alguém para dividir (acredite, meio lanche já serve como um almoço completo). Dependendo do recheio, a pizza é muito barata, também, embora eles não façam delivery. A Alemã fica na avenida 1 (a da saída da FEEC). É bom lembrar que eles servem café-da-manhã das 7h até às 13h (mas a padaria só fecha às 22h), então é uma boa pedida para se você não quiser almoçar ou para sábado e domingo, acordar tarde e tomar um café da manhã para valer pelo almoço.

Na estrada da Rhodia\footnote{Estrada da Rhodia é o outro nome da Avenida Albino J. B. Oliveira, uma vez que bem ao final da estrada, já no município de Paulínia, localiza-se uma planta industrial da Rhodia.}, próximo à entrada da Cidade Universitária II, há a Paneteria Di Capri, que tem um pão francês muito bom (a um preço legal) e também muita variedade (incluindo tortas e lanches). Além disso você também pode tomar seu café da manhã lá, pois como quase toda padaria eles também oferecem um cardápio bom para logo cedo. Se você estiver com bastante apetite, de sexta a domingo eles servem um buffet de café-da-manhã com muitas opções e a um preço fixo (em torno de R\$15). Na hora do almoço também são preparados alguns pratos (para comer no local e para levar) e também há um esquema onde você pede um grelhado e tem acesso livre a um balcão com saladas e outras coisas, como petiscos. À noite eles servem pizzas e também há o esquema do grelhado, exceto no inverno, quando eles servem um buffet de sopas.

Já se você está na Unicamp e quer uma padaria, a dica é a Padaria da FEA (fica próxima à Cantina da Mecânica). Lá eles tem pães, doces e bolos. Com uma diferença: há produtos especiais, como pão de queijo com linhaça ou alho e pão francês com soja. Mas não se assuste: por mais estranho que pareçam, os produtos de lá são muito bons! E não deixe para ir lá depois das aulas, pois a Padaria da FEA fecha às 17h.

\section*{E no fim de semana?}
\addcontentsline{toc}{section}{E no fim de semana?}

Nos fins de semana, nem o Bandejão nem quase nenhuma cantina da Unicamp abrem (e as que abrem só o fazem no sábado). Você vai ter que se virar fora da Unicamp. Na avenida 1 e proximidades ficam abertos o Terraço, o Bardana (apenas no sábado) e a Padaria Alemã já citados, além de vários restaurantes próximos à Alemã. Na avenida 2 tem o Aulus (mais caro no sábado que durante a semana; domingo, então, mais ainda, mas costuma ter camarão à milanesa;  porém a marmita tem opções de carne e acompanhamentos (peça patachu), é grande e não é cara como o self-service, 11 reais) e o Yaki-Ten (buffet de comida japonesa por pessoa). No centro de Barão não faltam opções. Tem (indo da entrada de Barão pela estrada da Rhodia) o Estância Grill, o Solar dos Pampas, o Estância d’Oliveira, o Vila Ré, o Ki-Pizza, o restaurante Baroneza, o Salsinha e Cebolinha (preço muito bom), o Pão de Açúcar, o McDonald’s, o Burger King e alguns restaurantes no Tilli Center (a dica é o Subway, por menos de 10 reais você pode se alimentar). Na avenida Santa Isabel e adjacências tem o Cronópio (numa rua paralela à Santa Isabel), raizes zen (culinária vegetariana), o Hot Dog Central e as Pizzarias Sapore Pizza e Pizza Fiori. Perto da moradia tem a Tonha (Canto do Acarajé) e o Kalunga Lanches. Por fim, próximo à padaria Di Capri, há alguns restaurantes mais caros, como a Romana (serviço parecido com o da Di Capri, porém um bocado mais cara), Pizzaria Gregória, o TBONE (eles também têm marmitex), o Greg Burgers (o hambúrguer e o milk-shake são excelentes), o Tábua das Marés e o Morena-flor.

\subsection*{Entrega em domicílio e comida para madrugadas}
\addcontentsline{toc}{subsection}{Entrega em domicílio e comida para madrugadas}

\begin{itemize}

\item \textbf{Bardana:}
\begin{itemize}
\item Endereço: Avenida Dr. Romeu Tortima, 1500
\item Telefone: (19) 3289-9073
\end{itemize}

\item \textbf{Casa da Moqueca:}
\begin{itemize}
\item Endereço: Rua Maria Ferreira Antunes, 123
\item Telefone: (19) 3289-3131
\item Obs.: Prato mais caro, mas serve duas pessoas.
\end{itemize}

\item \textbf{Ginza Bar:}
\begin{itemize}
\item Endereço: Rua Roxo Moreira, 1768
\item Telefone: (19) 3289-9281
\end{itemize}

\item \textbf{Marmitex Tia Rita:}
\begin{itemize}
\item Telefone: (19) 3249-2899
\item Obs.: Entrega em casa, é bom e barato.
\end{itemize}

\item \textbf{Marmitex Hailton:}
\begin{itemize}
\item Telefone: (19) 3249-0153
\item Obs.: Entrega em casa, é bom e barato.
\end{itemize}

\item \textbf{Makis Place:}
\begin{itemize}
\item Endereço: Avenida Albino J. B. de Oliveira, 976
\item Telefone: (19) 3367-3077
\end{itemize}

\item \textbf{McDonald's:}
\begin{itemize}
\item Telefone: (19) 3289-5840
\item Endereço: Avenida Albino J. B. de Oliveira, 1430
\item Obs.: Entrega das 11h às 23h.
\end{itemize}

\item \textbf{Restaurante Baroneza:}
\begin{itemize}
\item Endereço: Rua Benedito Alves Aranha, 44 - Centro de Barão Geraldo
\item Telefone: (19) 3289-9087
\item Site: \url{restaurantebaronesa.com.br}
\end{itemize}

\item \textbf{Subway:}
\begin{itemize}
\item Endereço: Avenida Albino J. B. de Oliveira, 1556
\item Telefone: (19) 3201-8411 / (19) 3201-8410
\item Site: \url{subdelivery.com.br}
\item Obs.: Entrega na região das avenidas 1 e 2.
\end{itemize}

\item \textbf{Terraço:}
\begin{itemize}
\item Endereço: Rua Roxo Moreira, 1344
\item Telefone: (19) 3289-7920
\end{itemize}

\item \textbf{TBONE Steak Bar:}
\begin{itemize}
\item Endereço: Rua Maria Tereza Dias da Silva, 700
\item Telefone: (19) 3289-0485
\end{itemize}

\item \textbf{Hot-dog Independência:}
\begin{itemize}
\item Endereço: Rua Angela Signol Grigol, 742
\item Telefone: (19) 3289-8805
\item Obs.: Tem vários tipos de hot-dogs (com catupiry, com cheddar, com frango...) e tem preços menores que os do Rod Burguers. O único problema é que eles cobram taxa de entrega para um lanche e fecham à meia-noite.
\end{itemize}

\item \textbf{Kalunga Lanches:}
\begin{itemize}
\item Endereço: Rua Sebastião Bonomi, 40. 
\item Telefone: (19) 3289-5236
\item Obs.: Localiza-se perto da moradia e ficam abertos até altas horas. Eles não entregam, mas você pode adiantar um lanche para viagem ligando nesse número.
\end{itemize}

\item \textbf{Lanchão \& Cia:}
\begin{itemize}
\item Endereço: Avenida Albino J. B. de Oliveira, 1214
\item Telefone: (19) 3289-3665
\end{itemize}

\item \textbf{Nadog's - Hot-dog do Nado:}
\begin{itemize}
\item Telefone: (19) 3029-2270
\end{itemize}

\item \textbf{Ponto 1:}
\begin{itemize}
\item Endereço: Rua Eduardo Modesto, 54
\item Telefone: (19) 3289-2378
\item Site: \url{www.ponto1bar.com}
\end{itemize}

\item \textbf{Mega Sandubão:}
\begin{itemize}
\item Endereço: Avenida Albino J. B. Oliveira, 2287
\item Telefone: (19) 3288-0204
\item Obs.: Entregam até meia-noite.
\end{itemize}

\item \textbf{Barão das Pizzas:}
\begin{itemize}
\item Endereço: Rua Agostinho Pattaro, 187
\item Telefone: (19) 3249-1630
\end{itemize}

\item \textbf{Pizza Fiori:}
\begin{itemize}
\item Endereço: Avenida Santa Isabel, 405
\item Telefone: (19) 3289-3514
\end{itemize}

\item \textbf{Pizza Show:}
\begin{itemize}
\item Telefone: (19) 3324-7480
\end{itemize}

\item \textbf{Sapore Pizza:}
\begin{itemize}
\item Telefone: (19) 3289-0228
\item Obs.: Preço bom. Entregam até meia-noite. 
\end{itemize}

\item \textbf{Super Mega Pizza:}
\begin{itemize}
\item Endereço: Rua Francisca Resende Merciai, 125B
\item Telefone: (19) 3288-0606 / (19) 3288-0608
\item Site: \url{www.supermegapizza.com}
\end{itemize}

\item \textbf{China In Box:}
\begin{itemize}
\item Endereço: Rua Romualdo Andreazzi, 333 - Jd. Trevo
\item Telefone: (19) 3254-5601
\end{itemize}

\item \textbf{Habib's:}
\begin{itemize}
\item Telefone: 0800-778-2828
\item Obs.: Não entrega em Barão.
\end{itemize}

\item \textbf{Pastelaria Oba-Oba:}
\begin{itemize}
\item Endereço: Rua Benedito Alves Aranha, 115
\item Telefone: (19) 3249-1908
\end{itemize}

\item \textbf{Terra Nova Pizzaria:}
\begin{itemize}
\item Telefone: (19) 3289-4072
\end{itemize}

\item \textbf{Barraquinhas:}
\begin{itemize}
\item Há várias barraquinhas de hot-dog no centro de Barão e perto da moradia. Destaque para o dog do terminal, o Hot Dog Central, o Pedrogue e o dogão da moradia. Se você quiser um lanche, uma boa pedida é o ``Star Tresh'' (Raimundão ou Guarujá, chame como você quiser), que fica perto do balão da avenida 2 e costuma ficar aberto até altas horas. Perto da Unicamp, ao lado do posto Ipiranga que fica na avenida 1 também tem um dog prensado muito bom e barato.
\end{itemize}

\end{itemize}

\section*{Gastronomia}
\addcontentsline{toc}{section}{Gastronomia}

\subsection*{Restaurantes}
\addcontentsline{toc}{subsection}{Restaurantes}

\begin{itemize}

\item \textbf{Aulus VideoBar \& Restaurant:}
\begin{itemize}
\item A comida de lá é muito boa, só que é muito caro também (especialmente no final de semana), exceto pelo marmitex. Um ambiente diferente, com bicicletas e ferroramas no teto, por exemplo.
\item Endereço: Avenida Prof. Atílio Martini, 939
\item Telefone: (19) 3289-4453
\item Site: \url{www.aulus.com.br}
\end{itemize}

\item \textbf{Batataria Suiça:}
\begin{itemize}
\item Do lado do Mega Sandubão, serve batatas recheadas bem diferentes. É um pouco caro, mas vale a pena conferir. Uma dica é que às terças-feiras você compra uma batata, mas recebe duas.
\item Endereço: Avenida Albino J. B. Oliveira - Praça José Geraldi, a 50m do posto Esso.
\item Telefone: (19) 3201-1174
\item Site: \url{www.battataria.com.br}
\end{itemize}

\item \textbf{Boi Falô:}
\begin{itemize}
\item O restaurante é uma rancho, com comida típica do interior. É excelente, mas um pouco caro (cerca de R\$35,00 por pessoa), um lugar perfeito para levar a família quando eles vêm te visitar (e pagam o almoço!). Abre apenas nos almoços de sábado e domingo.
\item Endereço: Rua do Sol, 600
\item Telefone: (19) 3287-6342
\end{itemize}

\item \textbf{Estância Grill:}
\begin{itemize}
\item Logo na entrada de Barão. Tem rodízios de carne e de pizza à noite. 
\item Endereço: Avenida Albino J. B. Oliveira, 271
\item Telefone: (19) 3289-8697 / (19) 3289-6055 / (19) 3289-1511
\end{itemize}

\item \textbf{La Salamandra:}
\begin{itemize}
\item Restaurante mexicano, localizado ao lado do Makis Place. Comida boa e preço compatível, ele tem uma barraquinha na feirinha do CB, às quartas.
\item Endereço: Avenida Albino J. B. Oliveira, 998
\item Telefone: (19) 3289-2011 / (19) 9277-4340
\item Site: \url{www.portalbaraogeraldo.com.br/anunciantes/la-salamandra-culinaria-mexicana-}
\end{itemize}

\item \textbf{Makis Place:}
\begin{itemize}
\item Temakeria próxima ao terminal.
\item Endereço: Avenida Albino J. B. Oliveira, 976
\item Telefone: (19) 3367-3077 
\item Site: \url{www.makis.com.br}
\end{itemize}

\item \textbf{Solar dos Pampas:}
\begin{itemize}
\item Fazem um esquema no aniversário das pessoas que sai por cerca de R\$18,00 com rodízio, cerveja, refrigerante, buffet, sorvete e pinga a vontade. Ao lado da Estância d'Oliveira.
\item Endereço: Avenida Dr. Romeu Tortima, 165
\item Telefone: (19) 3289-1484 / (19) 3289-7869
\end{itemize}

\item \textbf{Temakeria:}
\begin{itemize}
\item Vende só temaki e bebidas. O horário de funcionamento é bastante conveniente.
\item Horário de funcionamento: domingo a terça das 11h30 às 0h, quarta a sábado das 11h30 às 6h
\item Endereço: Avenida Dr. Romeu Tortima, 1259 (relativamente próximo à Unicamp, um pouco pra cima do Bardana)
\item Telefone: (19) 3289-0802
\item Site: \url{www.tmkr.com.br}
\end{itemize}

\item \textbf{Estância d'Oliveira:}
\begin{itemize}
\item Rodízio de massas perto do terminal. Bom e não é caro. Antigo Universo das Massas.
\item Endereço: Avenida Albino J. B. Oliveira, 576
\item Telefone: (19) 3289-5369
\end{itemize}

\item \textbf{Vila Ré - Pizza:}
\begin{itemize}
\item Pizzaria próxima do terminal e do supermercado Dalben. Tem alguns sabores diferentes, as pizzas são boas e o preço não é alto. Possui serviço de entrega das 18h às 23h.
\item Endereço: Avenida Albino J. B. Oliveira, 658
\item Telefone: (19) 3289-0319
\end{itemize}

\end{itemize}

\subsection*{Lanches}
\addcontentsline{toc}{subsection}{Lanches}

\begin{itemize}

\item \textbf{Açaizeiro Brasil:}
\begin{itemize}
\item Serve um açaí muito bom e vários tipos de comidas mais leves, como lanches naturais, crepes e saladas, além de vários sucos. O preço não é caro e a comida é boa.
\item Endereço: Avenida Santa Isabel, 518
\item Telefone: (19) 3365-6555
\item Site: \url{www.portalbaraogeraldo.com.br/anunciantes/acaizeiro-brasil}
\end{itemize}

\item \textbf{Burger King:}
\begin{itemize}
\item Endereço: Avenida Albino J. B. de Oliveira, 1000
\end{itemize}

\item  \textbf{Fran's Café:}
\begin{itemize}
\item Cafeteria localizada no Tilli Center. Vende lanches, cafés, doces, salgados e bebidas (quentes ou geladas). Fazem também cafés da manhã. Mas é um pouco caro.
\item Endereço: Avenida Albino J. B. de Oliveira, 1600
\end{itemize}

\item \textbf{Greg Burguers:}
\begin{itemize}
\item Uma lanchenete muito boa, mas também relativamente cara. Uma das especialidades lá é o milk-shake (realmente muito bom). Fica na estrada da Rhodia (na esquina da Paneteria Di Capri).
\item Endereço: Rua Maria Tereza Dias da Silva, 664
\item Telefone: (19) 3289-6400
\item Site: \url{www.gregburgers.com.br}
\end{itemize}

\item \textbf{Kalunga Lanches:}
\begin{itemize}
\item Destaque para o caldinho de feijão. 
\item Endereço: Rua Sebastião Bonomi, 40. 
\item Telefone: (19) 3289-5236
\end{itemize}

\item \textbf{Lanchão \& Cia:}
\begin{itemize}
\item Um dos melhores lanches de Campinas (quiçá o melhor) chega a Barão. Os lanches geralmente são grandes e muito bons, e os preços são compatíveis com a qualidade e quantidade. Eles servem no carro se você preferir, com uma bandeja que fica presa no vidro. Destaque para a batata frita, feita de uma forma muito diferente, extremamente crocante e quase cremosa por dentro.
\item Endereço: Avenida Albino J. B. Oliveira, 1214
\item Telefone: (19) 3289-3665
\item Site: \url{www.lanchaoecia.com.br}
\end{itemize}

\item \textbf{McDonald's:}
\begin{itemize}
\item Telefone: (19) 3289-5840
\item Endereço: Av. Albino J. B. de Oliveira, 1430
\end{itemize}

\item \textbf{Mega Sandubão:}
\begin{itemize}
\item Lanchonete localizada na estrada da Rhodia e entrega lanches até a meia noite. Tem tradição de ter preços caros, por isso não se estranhe. Muitos gostam bastante dessa lanchonete pela famosa maionese temperada que eles servem. Portanto, não se esqueça de pedi-la quando for comprar lanches.
\item Endereço: Avenida Albino J. B. Oliveira, 2287
\item Telefone: (19) 3288-0204
\end{itemize}

\item \textbf{Subway:}
\begin{itemize}
\item Vende dos mais variados tipos de lanches. Lanches muito bons, e não tão caros. Localiza-se no Tilli Center.
\item Endereço: Avenida Albino J. B. de Oliveira, 1556
\item Telefone: (19) 3201-8411 / (19) 3201-8410
\item Site: \url{subdelivery.com.br}
\end{itemize}

\end{itemize}

\subsection*{Bares}
\addcontentsline{toc}{subsection}{Bares}

\begin{itemize}

\item \textbf{Bagdá Café - Bar \& Esfiharia:}
\begin{itemize}
\item Esfihas boas, mas um pouco caras. Entregam em Barão (cardápio no site), mas em horários de pico costumam demorar um pouco. A música ambiente inclui música ao vivo e ritmos variados, desde a MPB ao Blues.
\item Endereço: Avenida Santa Isabel, 233
\item Telefone: (19) 3289-0541 / (19) 3289-1842
\item Site: \url{www.carlinoamaral.com.br/bagda} 
\end{itemize}

\item \textbf{Bar do Jair:}
\begin{itemize}
\item Famoso pela coxinha de carne seca. Tem música ao vivo, uma decoração boteco-chique e é bem conhecido em Campinas. Algumas vezes, é preciso fazer reservas com antecedência porque o bar lota. Fica a duas quadras do Ponto 1.
\item Endereço: Rua Eduardo Modesto, 212 (Vila Santa Isabel)
\item Telefone: (19) 3308-4825 / (19) 3326-2903
\item Site: \url{bardojair.com.br}
\end{itemize}

\item \textbf{Casa São Jorge:}
\begin{itemize}
\item Música ao vivo todas as noites, com boa variedade e uma pequena pista de dança.
\item Comida um pouco cara, mas muito boa.
\item Endereço: Avenida Santa Isabel, 655 (mais ou menos perto da Moradia)
\item Telefone: (19) 3249-1588
\item Site: \url{www.casasaojorgebar.com.br}
\end{itemize}

\item \textbf{Cachaçaria Água Doce:}
\begin{itemize}
\item Localizada na avenida 1, é um lugar frequentado por pessoas mais velhas, ótimo para comida e bebida (pinga, especialmente) mas é bem caro.
\item Endereço: Avenida Dr. Romeu Tortima, 593
\item Telefone: (19) 3289-5464
\item Site: \url{www.aguadoce.com.br}
\end{itemize}

\item \textbf{Empório do Nono:}
\begin{itemize}
\item Caro, péssimo serviço, mas tem um chopp muito bem tirado.
\item Endereço: Avenida Albino J. B. Oliveira, 1128 (quase em frente ao terminal).
\item Telefone: (19) 3289-0041
\item Site: \url{www.emporiodonono.com.br}
\end{itemize}

\item \textbf{Fernando's Bar:}
\begin{itemize}
\item Serve cerveja e lanches baratos e muito bons, principalmente por virem acompanhados de uma porção pequena de fritas.
\item Um lugar simples mas muito limpo e agradável, principalmente em relação ao atendimento.
\item Fecha às 23h de segunda a quinta e sábado, tem música ao vivo na sexta e por enquanto ainda não abre nos domingos.
\item No centro de Barão (perto do Santander).
\end{itemize}

\item \textbf{Quintal do Neto:}
\begin{itemize}
\item Tem cerveja a preços razoáveis, salgados (coxinha e quibe) grandes, e mesas de sinuca (de ficha e por hora).
\item Endereço: Avenida Dr. Romeu Tortima, 104 (no alto, próximo ao balão de entrada em Barão Geraldo).
\item Tefefone: (19) 3249-0104
\end{itemize}

\item \textbf{Marambar:}
\begin{itemize}
\item Possui bebidas, lanches, sucos e porções a preços razoáveis e tem promoção de cerveja dependendo do dia da semana.
\item Ambiente agradável, ao ar livre, muito próximo da Unicamp.
\item Bastante frequentado por computeiros e engenheiros em geral, além de muita gente de outros cursos.
\item Endereço: Rua Eurico Wanderley Morais Carvalho, 10 (muito próximo ao balão da avenida 1)
\item Telefone: (19) 3288-0996
\end{itemize}

\item \textbf{Rudá Bar:}
\begin{itemize}
\item É um bar agradável, com música ambiente, frequentemente ao vivo, e pista de dança.
\item Famoso principalmente pelo ``Forró das 6'', a partir das 18h de domingo.
\item Endereço: Avenida Santa Isabel, 490
\item Telefone: (19) 3249-3087
\item Site: \url{agendarudabar.blogspot.com}
\end{itemize}

\item \textbf{Sabor di Cevada (Bar da Coxinha):}
\begin{itemize}
\item Famoso pela coxinha (realmente boa). Vale a pena ir lá, mas é relativamente caro.
\item Localização: Perto da avenida Santa Isabel, na rua da Sapore Pizza.
\end{itemize}

\item \textbf{Star Clean:}
\begin{itemize}
\item É o bar mais próximo da entrada pela avenida 2 da Unicamp, e por isso está sempre cheio.
\item Um dos principais pontos de encontro depois da aula e tem um preço razoável.
\item Localização: No balão da avenida 2.
\end{itemize}


\item \textbf{Bar do Zé:}
\begin{itemize}
\item Barato, mas bem pequeno. Cerveja com desconto antes das 21h.
\item Endereço: Avenida Albino J. B. Oliveira, 1325 (bem em frente ao Pão de Açúcar).
\item Site: \url{www.obardoze.com.br}
\end{itemize}

\end{itemize}

\end{story}