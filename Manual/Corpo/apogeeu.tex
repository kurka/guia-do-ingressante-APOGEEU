% Este arquivo .tex será incluído no arquivo .tex principal. Não é preciso
% declarar nenhum cabeçalho

\begin{story}{A APOGEEU}

Antes de falar qualquer coisa, a primeira coisa que precisamos dizer é: \textbf{seja bem vindo à FEEC e parabéns por ter entrado na melhor faculdade de Engenharia Elétrica do país!}

Já que você já está aqui, não precisamos fazer propaganda das qualidades da nossa faculdade, do fato dela ser uma das raras instituições a ter nota 7 na CAPES (ou seja, nível de excelência internacional), da Unicamp ser a universidade mais bem avaliada pelo MEC, tanto na graduação quanto na pós-graduação, ou da nossa universidade ser responsável por cerca de 15\% da produção científica nacional.

Vamos aproveitar para falar aqui sobre a \textbf{APOGEEU}, a SUA associação de pós-graduandos a partir de agora!

Primeira pergunta: o que significa Associação de Pós-Graduandos (ou simplesmente APG)? Possivelmente quando você estava na graduação você tinha um centro acadêmico, que tinha como responsabilidade defender os alunos, representá-los junto à universidade, organizar atividades e outras coisas do tipo. Pois é, a APG tem essas funções também! A APOGEEU, por exemplo, organizou nos últimos tempos mini-cursos de estatística para experimentos, palestras, participou ativamente na reorganização do Programa de Estágio Docente da Unicamp e do estabelecimento de regras na faculdade para o acúmulo de bolsas de pós-graduação com outros rendimentos.

\section*{Trabalhar é preciso!}

A APOGEEU é atualmente uma das mais importantes associações de pós-graduandos do Brasil, de modo que tem tido papel fundamental na atual campanha pelo reajuste de bolsas de pós-graduação da CAPES e do CNPq, organizando manifestações e escrevendo artigos. Esse respeito adquirido é fruto de muito trabalho dos voluntários. Certamente poderia ser feito muito mais coisa, mas para isso é preciso de mais pessoas para trabalharem!

Nos últimos anos já realizamos muitas atividades, mas tínhamos planos de fazer muito mais! Queremos organizar cursos de redação científica, de línguas, de programação paralela em GPU e muitos outros. Queremos organizar mais palestras, mais discussões. Precisamos aumentar as ações pelo reajuste de bolsas (se continuarmos todos parados nada vai sair). Mas é claro, para isso precisamos de pessoas para ajudarem. Sem apoio não fazemos nada!

Então deixamos o convite para vocês todos participarem da APOGEEU, indo às nossas reuniões (que são realizadas todas as quartas, das 12h às 13h30) e/ou ajudando em atividades que interessem a vocês. Para ficarem informados sobre a APOGEEU, vocês podem se inscrever em nosso grupo de e-mails:

\begin{itemize}
\item \url{groups.google.com/group/apogeeu}
\end{itemize}

\section*{Oportunidades de crescimento pessoal}

Certamente que a parte principal de sua pós-graduação é no laboratório, mas tenha em mente que não é só isso e que há muitas outras formas de aprendizado que são importantes para você! Convênios de pesquisa, experiência de estágio docente, projetos de extensão universitária e atividades voluntárias também tem um papel importante na formação do professor/pesquisador que provavelmente você irá ser. A participação na associação de pós-graduandos e em atividades como representante discente junto à universidade são excelentes oportunidades para você fazer contatos importantes e aprender a ter um papel ativo onde quer que esteja, seja na academia, seja em uma empresa. Organização de eventos, discussões sobre a administração da universidade e maior conhecimento sobre as pesquisas desenvolvidas em outras partes da Unicamp são só alguns dos outros possíveis benefícios e que podem te dar um grande diferencial além de somente o título de mestre e doutor (que muitos outros no Brasil inteiro terão).

Assim, fica o incentivo a vocês buscarem outras atividades fora do seu mestrado/doutorado, para que o dinheiro que o cidadão do Estado de São Paulo investe em vocês possa valer muito mais! E fica também o convite para que essa atividade extra seja o trabalho junto à APOGEEU!

\section*{Onde encontrar a APOGEEU}

A sede da associação fica no prédio da pós graduação da FEEC, na sala PE27 (subindo um piso pelas escadas, entre no corredor à esquerda. Será a primeira porta à sua esquerda, antes dos armários). Para mais informações quanto à APOGEEU, incluindo o horário das reuniões e as normas para uso dos armários, visite o site:

\begin{itemize}
\item \url{www.apogeeu.fee.unicamp.br}
\end{itemize}

E, como somos modernos, temos Twitter -- \url{twitter.com/apogeeu} --, Facebook -- \url{facebook.com/APOGEEU} -- e Google+ -- \url{goo.gl/WQ9fO}. Precisando de qualquer coisa, você pode entrar em contato conosco pelo e-mail \url{apogeeu@fee.unicamp.br}.

\end{story}

