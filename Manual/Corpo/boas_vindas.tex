% Este arquivo .tex será incluído no arquivo .tex principal. Não é preciso
% declarar nenhum cabeçalho

\setcounter{page}{1}

\begin{story}{Boas vindas}

Caro colega,

Parabéns, você acabou de ingressar em um dos poucos programas de pós-graduação em Engenharia Elétrica classificados pela CAPES como curso de excelência máxima (nota 7)! Em breve você irá perceber que não é apenas a estrutura e o nível do corpo docente que fazem uma pós-graduação desse nível, mas também a qualidade e dedicação dos seus alunos. Mas não se assuste! Apesar de todo o suor que você terá pela frente, você verá que ainda é possível conciliar trabalho firme nas aulas e na sua pesquisa com a vida social em Barão Geraldo e Campinas.

Este manual\footnote{Esse manual também pode ser encontrado em nossa página na internet, no endereço \url{www.apogeeu.fee.unicamp.br/manual-do-ingressante}.} foi organizado pela APOGEEU (Associação de Pós-Graduandos da Faculdade de Engenharia Elétrica e de Computação da Unicamp) com base no manual do Centro Acadêmico da Computação (CACo) -- \url{http://www.caco.ic.unicamp.br} --, escrito por diversos alunos que contribuíram com informações que serão especialmente úteis para você, ingressante! Assim, esperamos que vocês nos mandem informações sobre erros encontrados, informações sobre restaurantes novos ou que fecharam e por aí vai! Nosso e-mail é \url{apogeeu@fee.unicamp.br}.

Onde comer? Onde estudar? Onde morar? Tudo isso são dúvidas comuns, que aqui tentamos ajudar a resolver. Não há respostas prontas, cada um tem suas preferências, mas a gente dá uma mão.

O que é APG? E CPG? E CCPG? E RPG? Como eu faço para pegar uma bolsa? A gente também tenta responder todas essas perguntas. E também damos algumas dicas de onde comprar coisas, onde se divertir e alguns telefones úteis.

Já demos os parabéns, então agora vamos para o que importa: dar umas dicas para os seus primeiros passos na pós-graduação da Unicamp e apresentar a APOGEEU!

\section*{Mensagem do diretor da FEEC}
\addcontentsline{toc}{section}{Mensagem do diretor da FEEC}

Prezados pós-graduandos da FEEC,

É com grande satisfação que lhes damos boas-vindas à comunidade da Faculdade de Engenharia Elétrica e de Computação. Somos quase 100 docentes, quase 2000 estudantes, entre graduação e pós-graduação e cerca de 50 colaboradores que possibilitam o funcionamento de nossa escola. No orgulhamos de ser o melhor programa de pós-graduação em Engenharia Elétrica do Brasil, o que foi conseguido pelo mérito conjunto de nossos estudantes e professores.

A pós-graduação é mais um passo em um processo de educação que, mais do que nunca, tem que ser contínuo, permanente. O conhecimento é uma riqueza que não ocupa espaço, não tem massa, mas tem peso: o peso da responsabilidade. Estudar em uma escola pública, financiada com os impostos pagos pela sociedade paulista e brasileira, acresce ainda mais responsabilidade neste processo.

Os saberes de cada um estão em proporção direta com o que se pode esperar de sua atuação profissional, social, política. Assim, desejamos que a nova jornada que se inicia na vida de cada um de vocês lhes traga uma expansão de seus conhecimentos, seja em profundidade, seja em amplitude. Que se estabeleçam novas amizades e que estas perdurem no futuro. Que tenham uma atitude crítica perante nossa escola, o que nos permitirá melhorar continuamente, assim como ante o país e o mundo.

Daqui a algum tempo, terminada esta nova jornada, esperamos que as lembranças destes tempos que estão se iniciando sejam as melhores possíveis e que levar consigo um diploma da Unicamp seja um orgulho e um compromisso com um mundo melhor.

Bem-vindos à FEEC e à Unicamp!

\begin{flushright}
Prof. José Antenor Pomilio \\
Diretor da FEEC/Unicamp
\end{flushright}

\section*{Mensagem do coordenador de pós-graduação}
\addcontentsline{toc}{section}{Mensagem do coordenador de pós-graduação}

Caros alunos e alunas ingressantes,

Sejam muito bem-vindos ao Programa de Pós-Graduação em Engenharia Elétrica da Faculdade de Engenharia Elétrica e de Computação (FEEC) da UNICAMP. Vocês agora fazem parte de um dos melhores programas de pósgraduação do país nessa área, agraciado com o conceito máximo (nota 7) pela Coordenação de Aperfeiçoamento de Pessoal de Nível Superior (CAPES), órgão do governo brasileiro que certifica e avalia os cursos do ensino superior no Brasil. Somos um dos programas mais antigos de mestrado e doutorado em Engenharia Elétrica no Brasil, que formou mais de 2100 mestres, cerca de 950 doutores e que possui inserção importante tanto no cenário nacional quanto no internacional.

Nessa nova fase de formação e amadurecimento profissional, vocês serão bastante exigidos em termos de desempenho acadêmico e dedicação. Por um lado, a CPG-FEEC fará todos os esforços para oferecer um ambiente de alta qualidade para os estudos, com salas de aula modernas e bem aparelhadas, laboratórios bem estruturados e com equipamentos de ponta, e professores dedicados, qualificados e com total envolvimento com orientação, docência e pesquisa. Em contrapartida, serão cobrados dos alunos dedicação e desempenho compatíveis com um curso de pós-graduação do mais alto nível, para que a formação de mestres e doutores na FEEC seja acompanhada pela pesquisa de excelente qualidade, voltada para a melhoria e o desenvolvimento da sociedade em sua totalidade. Cada mestrado e doutorado concluído agrega valores e benefícios para orientados e orientadores, para colegas discentes e docentes, para a FEEC e para a UNICAMP como um todo, proporcionando a gratificante sensação de dever cumprido em relação à sociedade brasileira, que financia o ensino e a pesquisa nas universidades públicas.

É muito importante para todos vocês conhecer as regras dos cursos da FEEC, da Unicamp, os direitos e deveres dos alunos de pós-graduação. Não temos na FEEC um exame de ingresso para a pós-graduação, mas temos critérios para permanência no programa. Não temos uma cota de bolsas que cubra todas as nossas necessidades, mas existem outros mecanismos para conseguir o necessário apoio financeiro ao longo dos estudos. Como pleitear outros tipos de bolsa? Quantos créditos em disciplinas são necessários para o mestrado e para o doutorado? Quais os requisitos para o exame de qualificação? Quais as condições mínimas para poder defender uma dissertação ou tese? Essas e outras perguntas são respondidas na legislação dos cursos de pós-graduação da UNICAMP, nos documentos como o Regimento Geral dos Cursos de Pós-graduação da UNICAMP, o Catálogo dos Cursos de Pósgraduação da FEEC, o Regulamento de Pós-graduação da FEEC. Além disso, temos o calendário da Unicamp, com as datas e os prazos importantes, e as Instruções CPG-FEEC que regulamentam a gestão interna dos nossos cursos. Todas as informações pertinentes podem ser acessadas a partir da página da Comissão de Pós-Graduação (CPG) da FEEC, em

\begin{itemize}
\item \url{www.fee.unicamp.br/cpg}
\end{itemize}

como por exemplo nos links Legislação, Alunos Regulares, Perguntas Frequentes (FAQ), etc. Além disso, vocês podem contar sempre com a equipe de funcionários da CPG-FEEC para dirimir qualquer dúvida e ajudá-los da melhor maneira possível.

Finalmente, gostaria de concluir expressando meus sinceros votos de muito sucesso para todos vocês.

\begin{flushright}
Prof. Pedro Luis Dias Peres
Coordenador de Pós-Graduação da FEEC/Unicamp
\end{flushright}

\end{story}