% Este arquivo .tex será incluído no arquivo .tex principal. Não é preciso
% declarar nenhum cabeçalho

\begin{story}{Grupos e comunidades}

\section*{Pró-Pós -- Movimento de Pós-Graduandos da Unicamp}
\addcontentsline{toc}{section}{Pró-Pós -- Movimento de Pós-Graduandos da Unicamp}

O Pró-Pós é um grupo que já existe há mais de cinco anos, tendo dentre seus fundadores a APOGEEU. Ele busca juntar Associações de Pós-Graduandos, Centros Acadêmicos que representam a pós-graduação, representantes discentes dos mestrandos e doutorandos e quaisquer outras pessoas que queiram organizar atividades para os alunos de pós-graduação da universidade ou discutir os rumos que a Unicamp (e o ensino superior e a pesquisa no Brasil) está tomando e como os alunos de pós-graduação se relacionam com esses rumos, sempre tendo como pedra fundamental o diálogo.

O Pró-Pós também foi o grupo mais ativo na organização de manifestações pelo reajuste das bolsas de pós-graduação da CAPES e do CNPq, que haviam ficarado por mais de 4 anos sem reajuste. Além disso, participamos ativamente na discussão de questões internas à universidade como, por exemplo, a reestruturação do Programa de Estágio Docente (PED), que teve várias modificações implementadas advindas de propostas discutidas dentro do Pró-Pós.

Se você quiser ajudar em alguma atividade, participar das discussões ou só ficar a par do que está acontecendo na Unicamp, se inscreva na lista de e-mail:

\begin{itemize}
\item Site: \url{groups.google.com/group/pos-unicamp}
\item E-mail: pos-unicamp@googlegroups.com
\end{itemize}

\section*{Rádio Muda}
\addcontentsline{toc}{section}{Rádio Muda}

Você provavelmente nunca viu nada do tipo na sua vida. Uma rádio na qual qualquer ser humano pode fazer o seu programa tranquilamente, sem burocracias (tendo espaço na grade de horários, lógico).

A Rádio Muda fica embaixo da caixa d'água (carinhosamente apelidada de Pau do Zeferino) que fica perto do Teatro de Arena, bem em frente à BC (Biblioteca Central).

Se você só quiser ouvir a muda, 88,5 no seu rádio (em Barão Geraldo ou Paulínia) ou pela Internet, através do site \url{muda.radiolivre.org}.

\section*{Curso Exato}
\addcontentsline{toc}{section}{Curso Exato}

O Curso Exato tem como principal objetivo auxiliar no aprendizado das disciplinas de matemática, física e química de forma direta e dinâmica, ou seja, tentar preencher com algum conhecimento as mentes ocas dos estudantes de ensino médio de Campinas. Tem como professores alunos de graduação Unicamp, orientados por docentes da instituição. O curso possui caráter comunitário, sendo totalmente gratuito. Conta com o apoio da pró-reitoria de extensão e assuntos comunitários (PREAC) e do curso pré-vestibular Cooperativa do Saber.

\begin{itemize}
\item Email: \url{curso.exato@gmail.com.br}
\item Site: \url{www.cursoexato.com.br}
\end{itemize}

\section*{Cursinho da Moradia}
\addcontentsline{toc}{section}{Cursinho da Moradia}

Pré-Vestibular Popular na Moradia Estudantil, busca propiciar a troca ativa de conhecimento, estimulando o senso crítico e o reconhecimento pelo estudante de sua condição social de exclusão, tal como a possibilidade de transformar a sociedade.

\begin{itemize}
\item Blog: \url{cursinhodamoradiaunicamp.blogspot.com}
\end{itemize}

\section*{Projeto Educacional Machado de Assis}
\addcontentsline{toc}{section}{Projeto Educacional Machado de Assis}

Trata-se de um projeto educacional que funciona no Instituto de Estudos da Linguagem da Unicamp. É destinado a estudantes de baixa renda e visa, além do ingresso destes nas universidades públicas, proporcionar uma formação crítica, inclusive sobre o próprio caráter excludente do vestibular e da universidade

\begin{itemize}
\item Blog: \url{www.cursinhomachadodeassis.wikispaces.com}
\end{itemize}

\section*{Sonha Barão}
\addcontentsline{toc}{section}{Sonha Barão}

Envolve os alunos da Unicamp no desenvolvimento de trabalhos sociais em Barão Geraldo, promovendo, assim, a integração da universidade com a comunidade e proporcionando a vivência da cidadania e a formação do homem integral.

\section*{Pastoral Universitária}
\addcontentsline{toc}{section}{Pastoral Universitária}

Grupo católico que se reúne semanalmente para estudar textos (bíblicos ou não), livros, documentos, aprofundar a fé e promover a integração e união de seus participantes. A Pastoral Universitária também organiza grupos de preparação para Primeira Comunhão e Crisma, além de duas Missas semanais e Grupos de Oração Universitários (GOUs). As reuniões da PU acontecem às quartas, em dois horários: 12h15 e 18h. As Missas são realizadas às quintas (12h15) e às terças (18h10). Os GOUs acontecem nas terças (12h15) e nas quintas (18h). O local é sempre o mesmo para todas as atividades: a sala PB04.

\begin{itemize}
\item E-mail: \url{pastoralunicamp@gmail.com}
\item Site: \url{sites.google.com/site/pastoralunicamp}
\end{itemize}

\section*{ABU - Aliança Bíblica Universitária}
\addcontentsline{toc}{section}{ABU - Aliança Bíblica Universitária}

Grupo evangélico não ligado a nenhuma denominação que organiza várias reuniões e grupos de discussões, filiado à Aliança Bíblica Universitária do Brasil (\url{www.abub.org.br}). Quaisquer dúvidas, entre no site: \url{www.abucampinas.org}, mande um e-mail para \url{contato@abucampinas.org} ou \url{abucamp_co@yahoogrupos.com.br} ou ainda ligue para (19) 3289-2823.

\end{story}